\documentclass[12pt, a4paper, openright, twoside]{abntex2}                          % modelo abntex2(memoir). Fonte 12, papel A4, twoside(ambos lados) e openright(anverso) para elementos pré textuais e novos chapters
\usepackage[T1]{fontenc}                                                            % fonte
\usepackage[utf8]{inputenc}                                                         % codificação do arquivo tex, caracteres utf8
\usepackage[brazil]{babel}                                                          % hifenização
% \usepackage{sectsty}                                                                % formatação de seções
% \sectionfont{\clearpage}                                                            % nova página para cada seção 
% \usepackage[a4paper,top=3cm,left=3cm,right=2cm,bottom=2cm]{geometry}                % bordas
\PassOptionsToPackage{hyphens}{url}\usepackage{hyperref}                            % formatação de url's e hyperlink neles
\usepackage[num,overcite]{abntex2cite}                                              % citações estilo ABNT, opção [alf] para citação (autor, data) e [num,overcite] para citação estilo [1]. https://linorg.usp.br/CTAN/macros/latex/contrib/abntex2/doc/abntex2cite.pdf
\citebrackets[]																		% estilo de colchetes para as fontes
% \usepackage[nottoc]{tocbibind}                                                      % opções do sumário, bibliografia no sumário, tirar autorreferência ao sumário 
\usepackage{graphicx}                                                               % imagens e gráficos graphviz
\usepackage{tikz}                                                                   % imagens tikz, usado com dot2tex
\usepackage{pgflibrarysnakes}                                                       % símbolos adicionais para imagens
\usepackage{pgflibraryarrows,pgflibraryshapes}                                      % símbolos adicionais para imagens
\usepackage{amsmath}                                                                % equações
\usepackage{mathtools}                                                              % mais símbolos matemáticos
\usepackage{amssymb}                                                                % símbolos, como flechas e outros
\usepackage{mathrsfs}                                                               % símbolos extras com comando \mathscr, ex: símbolos para transformadas F, Z e L
\usepackage{siunitx}                                                                % unidades SI
\usepackage{indentfirst}                                                            % identar primeira linha depois do comando \section ou \subsection
\usepackage{xcolor}                                                                 % definição de cores, usado na listagem de código
% \renewcommand{\arraystretch}{1.7}                                                   % separação vertical entre células em tabelas
\usepackage{pdfpages}                                                               % inclusão de pdf's

\usepackage{luacode}                                                                % macros para melhor execução de código lua, \luaexec e \begin{luacode} ... \end{luacode}. https://linorg.usp.br/CTAN/macros/luatex/latex/luacode/luacode.pdf
\usepackage{luapackageloader}                                                       % pra máquina lua procurar pacotes nos paths
\directlua{package.path = "/usr/share/lua/5.4/?.lua;" .. package.path }             % anexa o meu path do lua rocks pra pacotes no package.path, mudar se necessário

% comandos para usar no tabular de forma a dar tamanho para as colunas, util para dar espaço extra ou realizar quebra de texto dentro das células. Ex: \begin{tabular}{|cC{2cm}L{4cm}r|}
% left fixed width:
\newcolumntype{L}[1]{>{\raggedright\arraybackslash}p{#1}}
% center fixed width:
\newcolumntype{C}[1]{>{\centering\arraybackslash}p{#1}}
% flush right fixed width:
\newcolumntype{R}[1]{>{\raggedleft\arraybackslash}p{#1}}

% Listagem de código fonte
\usepackage{listings}
\definecolor{background_color}{HTML}{f8f8fd}
\definecolor{comment_color}{HTML}{6AAF19}
\definecolor{keyword_color}{HTML}{F92672}
\definecolor{ndkeyword_color}{HTML}{00FF00}
\definecolor{string_color}{HTML}{F25A00}
\definecolor{identifier_color}{HTML}{000000}
\definecolor{line_number_color}{HTML}{000000}
\lstdefinestyle{mystyle}{
    backgroundcolor=\color{background_color},   
    commentstyle=\color{comment_color},
    keywordstyle=\color{keyword_color},
    ndkeywordstyle=\color{ndkeyword_color},
    numberstyle=\tiny\color{line_number_color},
    stringstyle=\color{string_color},
    identifierstyle=\color{identifier_color},
    basicstyle=\ttfamily\footnotesize,
    breakatwhitespace=false,         
    breaklines=true,                 
    captionpos=b,                    
    keepspaces=true,                 
    numbers=left,                    
    numbersep=5pt,                  
    showspaces=false,                
    showstringspaces=false,
    showtabs=false,                  
    tabsize=2,
    captionpos=t,
    aboveskip=15pt,
    belowskip=15pt
}
\renewcommand\lstlistingname{Código}
\lstset{style=mystyle}

\renewcommand{\legend}[1]{\par Fonte: #1}											% definir prefixo para comando de legenda de fonte que fica em baixo de figuras e tabelas
\setFloatSpacing{1}																	% espaçamento entre linhas em elementos float(imagens, tabelas e outros) para a legenda e fonte 
\setlength{\afterchapskip}{1.5ex}													% espaçamento entre titulo do capitulo e o paragrafo seguinte de aproximadamente 1.5
\setlength{\beforesecskip}{1.5ex}													% espaçamento entre titulo da seção e o paragrafo anterior de aproximadamente 1.5
\setlength{\aftersecskip}{1.5ex}       												% espaçamento entre titulo da seção e o paragrafo seguinte de aproximadamente 1.5
\setlength{\beforesubsecskip}{1.5ex}   												% espaçamento entre titulo da sub seção e o paragrafo anterior de aproximadamente 1.5
\setlength{\aftersubsecskip}{1.5ex}    												% espaçamento entre titulo da sub seção e o paragrafo seguinte de aproximadamente 1.5
\setlength{\beforesubsubsecskip}{1.5ex}												% espaçamento entre titulo da sub sub seção e o paragrafo anterior de aproximadamente 1.5
\setlength{\aftersubsubsecskip}{1.5ex} 												% espaçamento entre titulo da sub sub seção e o paragrafo seguinte de aproximadamente 1.5

% ajustes para o espaçamento no sumário de 1.5
\setlength{\cftbeforechapterskip}{1.5ex}
\setlength{\cftbeforepartskip}{1.5ex}
\setlength{\cftbeforesectionskip}{1.5ex}
\setlength{\cftbeforesubsectionskip}{1.5ex}
\setlength{\cftbeforesubsubsectionskip}{1.5ex}

% ajustes para anexo maiúsculo no sumário, no caso de Sumario (TOC) especifico da ABNT-6027-2012 precisa desse debaixo pro anexo ficar minúsculo
% \makeatletter
% \ifthenelse{\boolean{ABNTEXsumario-abnt-6027-2012}}{
%   \settocpreprocessor{chapter}{%
%   \let\tempf@rtoc\f@rtoc%
%   \def\f@rtoc{%
%   \texorpdfstring{\tempf@rtoc}{\tempf@rtoc}}%
%   }
%   \settocpreprocessor{part}{%
%   \let\tempf@rtoc\f@rtoc%
%   \def\f@rtoc{%
%   \texorpdfstring{\tempf@rtoc}{\tempf@rtoc}}%
%   }
% }{}
% \makeatother

\renewcommand{\ABNTEXchapterfontsize}{\normalsize}         							% tamanho das fontes para \normalsize, tamanho 12
\renewcommand{\ABNTEXpartfontsize}{\normalsize}            							% tamanho das fontes para \normalsize, tamanho 12
\renewcommand{\ABNTEXsectionfontsize}{\normalsize}         							% tamanho das fontes para \normalsize, tamanho 12
\renewcommand{\ABNTEXsubsectionfontsize}{\normalsize}      							% tamanho das fontes para \normalsize, tamanho 12
\renewcommand{\ABNTEXsubsubsectionfontsize}{\normalsize}   							% tamanho das fontes para \normalsize, tamanho 12
\renewcommand{\ABNTEXsubsubsubsectionfontsize}{\normalsize}							% tamanho das fontes para \normalsize, tamanho 12

% % fontes dos títulos e seções
% % coloca negrito nas coisas corretas para o sumário também sair correto.
% % os títulos dos capítulos e seções tem que ser colocados maiúsculos manualmente. Preço a se pagar para que os apêndices e anexos não fiquem todos maiúsculos.
% % fontes padrões de part, chapter, section, subsection e subsubsection
% \renewcommand{\ABNTEXchapterfont}{\fontseries{bx}\fontshape{n}\selectfont}
% \renewcommand{\ABNTEXchapterfontsize}{\normalsize\selectfont}
% \renewcommand{\ABNTEXpartfont}{\fontseries{bx}\fontshape{n}\selectfont}
% \renewcommand{\ABNTEXpartfontsize}{\Huge\selectfont}
% \renewcommand{\ABNTEXsectionfont}{\fontseries{m}\fontshape{n}\selectfont}
% \renewcommand{\ABNTEXsectionfontsize}{\normalsize\selectfont}
% \renewcommand{\ABNTEXsubsectionfont}{\fontseries{b}\fontshape{n}\selectfont}
% \renewcommand{\ABNTEXsubsectionfontsize}{\normalsize\selectfont}
% \renewcommand{\ABNTEXsubsubsectionfont}{\fontseries{m}\fontshape{n}\selectfont}
% \renewcommand{\ABNTEXsubsubsectionfontsize}{\normalsize\selectfont}
% \renewcommand{\ABNTEXsubsubsubsectionfont}{\fontseries{m}\fontshape{it}\selectfont}
% \renewcommand{\ABNTEXsubsubsubsectionfontsize}{\normalsize\selectfont}

% % sumário
% % coloca negrito nas coisas corretas para o sumário também sair correto.
% \renewcommand{\cftpartfont}{\ABNTEXpartfont}
% \renewcommand{\cftpartpagefont}{\cftpartfont}

% \renewcommand{\cftchapterfont}{\ABNTEXchapterfont}
% \renewcommand{\cftchapterpagefont}{\cftchapterfont}

% \renewcommand{\cftsectionfont}{\ABNTEXsectionfont}
% \renewcommand{\cftsectionpagefont}{\cftsectionfont}

% \renewcommand{\cftsubsectionfont}{\ABNTEXsubsectionfont}
% \renewcommand{\cftsubsectionpagefont}{\cftsubsectionfont}

% \renewcommand{\cftsubsubsectionfont}{\ABNTEXsubsubsectionfont}
% \renewcommand{\cftsubsubsectionpagefont}{\cftsubsubsectionfont}

% \renewcommand{\cftparagraphfont}{\ABNTEXsubsubsubsectionfont}
% \renewcommand{\cftparagraphpagefont}{\cftparagraphfont}
